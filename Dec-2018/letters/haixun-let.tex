\documentclass[11pt]{article} 

\usepackage{deauthor,times,graphicx}
%\usepackage{url}
\usepackage{hyperref}

\begin{document}


\subsection*{Thank You, David!}

I know I represent the readers, the associate editors, and also the
broad database community when I say we are extremely grateful to David
Lomet for his distinguished and dedicated service as the
Editor-in-Chief of the Data Engineering Bulletin for the last 26
years.

Since its launch in 1977, the Bulletin has produced a total of 154
issues. Reading through the topics of the past issues that spanned
more than four decades makes me feel nothing short of amazing. They
show not just how far the database research has come, but to a certain
extent, how much the entire field of computer science and the IT
industry have evolved. While important topics never fail to arise in
the Bulletin in a timely fashion, it is also interesting to observe in
the 154 issues many recurring topics, including query optimization,
spatial and temporal data management, data integration, etc. It proves
that the database research has a solid foundation that supports many
new applications, and at the same time, it demonstrates that the
database research is constantly reinventing itself to meet the
challenges of the time. What the Bulletin has faithfully documented
over the last 42 years is nothing else but this amazing effort.

Among the 154 issues since the launch of the Bulletin, David had been
the Editor-in-Chief for 103 of them. This itself is a phenomenal
record worth an extra-special celebration. But more importantly, David
shaped the discussions and the topics in the long history of the
Bulletin.  I had the honor to work with David in 2016 and 2017 when I
served as the associate editor for two Bulletin issues. What was most
appealing to me was the opportunity of working with the top experts on
a topic that I am passionate about. The Bulletin is truly unique in
this aspect.

I understand the responsibility and the expectation of the
Editor-in-Chief, especially after David set such a great example in
the last 26 years. I thank David and the associate editors for their
trust, and I look forward to working with authors, readers, and the
database community on the future issues of the Data Engineering
Bulletin.

\subsection*{The Current Issue}

Machine learning is changing the world.  From time to time, we are
amazed at what a few dozen lines of python code can achieve (e.g.,
using PyTorch, we can create a simple GAN in under 50 lines of code).
However, for many real-life machine learning tasks, the challenges lie
beyond the dozen lines of code that construct a neural network
architecture. For example, hyperparameter tuning is still considered a
``dark art,'' and having a platform that supports parallel tuning is
important for training a model effectively and efficiently. Model
training is just one component in the life cycle of creating a machine
learning solution. Every component, ranging from data preprocessing to
inferencing, requires just as much support on the system and
infrastructure level.

Joseph Gonzalez put together an exciting issue on the life cycle of
machine learning. The papers he selected focus on systems that help
manage the process of machine learning or resources used in machine
learning. They highlight the importance of building such supporting
systems, especially for production machine learning platforms.

\end{document}

