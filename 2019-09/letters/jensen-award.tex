\documentclass[11pt]{article}
\usepackage{deauthor,times}

\begin{document}

%\noindent
%\textbf{\Large Letter from the 2019 IEEE TCDE Impact Award Winner}
%\newline

% \title{Letter from the 2019 IEEE TCDE Impact Award Winner}
% \maketitle
I was very happy and humbled to receive this year's TCDE Impact Award, with the citation ``for contributions to spatial, temporal, and spatio-temporal data
management.'' I would like to thank those who nominated me as well as the awards'd committee. Conducting research is very much a social, or collaborative, activity, and I have worked with many excellent colleagues on the three topics mentioned in the citation, and they deserve most of the credit for the results that I have contributed to achieving. I will mention some of them as I cover aspects of my research journey. I started out working on temporal databases and then later transitioned to working on spatial and spatio-temporal databases. To achieve some degree of brevity, I will offer an account of only some of the activities related to temporal data management. I thus start at the very beginning of my academic life.

\paragraph{The Early Years---Ph.D.\ Studies} I received my M.Sc.\ degree in computer science from Aalborg University in 1988.  At that time, the M.Sc.\ study had a formal duration of five and a half years and included two B.Sc.\ degrees (in my case, in Mathematics and Computer Science). The last half year was devoted to the M.Sc.\ thesis, but the mindset at the time was that you were not serious if you spent less than a year. Thus, having received the M.Sc.\ degree after six years of study, I received a scholarship to go and study for a Ph.D.\ for two and a half years anywhere in the world. All I needed to do was to write a thesis---the course requirements were already satisfied.

In early September 1988, I then arrived at Dulles Airport. My M.Sc.\ supervisor, Lars Mathiassen, now a professor at Georgia State University, had recommended that I study under the direction of Leo Mark, then a young faculty member at the University of Maryland. I still remember driving with Leo from Dulles to his house in the late evening with all the windows open in his (by Danish standards) huge and very American Chevy. An exciting journey had started.

A November 25, 1988 plan gave the following working title for my thesis: ``A By-Relation Implemented Object Oriented Data Model Supporting Efficient Storage and Retrieval of Versions of Complex Objects in Engineering Applications.'' I started out looking at the versioning aspect, and this led to studies of support for transaction time, which I viewed as an ideal foundation for fine-grained version support. The eventual title of the thesis was ``Towards the Realization of Transaction Time Database Systems,'' and I had become interested in temporal databases.

\paragraph{The Pursuit of Industrial Impact} Having completed the Ph.D.\ studies and defended the thesis back in Denmark in January 1991, I packed up my car in Greenbelt, MD and drove cross-country to Tucson, AZ, where I was to work with the most visible temporal database researcher, Rick Snodgrass, then a young faculty member at the University of Arizona. I had received a faculty position at Aalborg University that allowed me to spend my first semester with Rick. Our interests matched very well, and we got off to a very good start. This turned into three more sabbaticals, in 1992, 1994, and 1999, where I also got the opportunity to work with Rick's students, Curtis Dyreson, Nick Kline, and Mike Soo.

The 1990s were exciting times in temporal databases. The field had witnessed a proliferation of temporal data models and query languages, almost to the point of each researcher having their own model and language. It was felt that this blocked industrial impact, and initiatives were taken to achieve a consensus temporal data model and query language. This resulted in the TSQL2 query language, which was designed by an 18-person committee led by Rick. 

Pursuing the goal of achieving industrial impact, Rick subsequently was the main force behind attempts to standardize TSQL2. This turned out to be a difficult process, in part due to politics and a variety of interests, but we also made technical progress. Specifically, we learned that the TSQL2 design approach did not scale well: Adding support for some temporal functionality to SQL worked fine, but adding comprehensive support following the TSQL2 approach was not pretty. While SQL is not a pretty language in the first place in terms of design, the TSQL2 approach yielded a result that was uglier than we would have liked. Something different was needed. As we were making these revelations, Michael B\"{o}hlen joined the University of Arizona as a postdoc. He had worked on an approach to language design that inspired the introduction of so-called statement modifiers into TSLQ2. The idea is that many temporal queries can be expressed intuitively and unambiguously as a single-state, non-temporal (and easy-to-formulate) SQL query that is then performed, as specified by a statement modifier, on all states of a temporal relation, after which the results are combined into a temporal relation. So a temporal query could then be formulated by a non-temporal query prefixed by some modifiers. A careful design based on this approach was introduced into standards proposals, and an ``academic'' version called ATSQL was also designed and documented in a TODS 2000 paper titled ``Temporal Statement Modifiers.''

In parallel with the above, I also worked on a range of other subjects in temporal databases, including database design, covering logical and conceptual temporal database design; data model and query language design aspects; support for the notion of ``now'' and for data aging; indexing; implementation of temporal algebra operators; query optimization; and architectures for implementing temporal query language support. I worked with five of my first six Ph.D.\ students on these topics: Kristian Torp, Heidi Gregersen, Dieter Pfoser, Janne Skyt, and Giedrius Slivinskas.

\paragraph{The Recent Years} While spatial and spatio-temporal databases started to take over as my main activity around year 2000, I have continued to maintain an interest in temporal databases. Following his postdoc at Arizona, Mike joined the faculty at Aalborg University. He later moved to the Free University of Bozen-Bolzano and he is now back home in Switzerland, at the University of Zurich. I have been fortunate to be able to continue to work on temporal databases with Mike, Hans Gamper from Bolzano, and most recently Anton Dign\"{o}s, as a Ph.D.\ student at Zurich and now as a faculty member at Bolzano. A key goal was to achieve an implementation of ATSQL. With other colleagues, we looked at many options, but it took until 2016, i.e., 16 years, before we had solid results. In particular, Anton's Ph.D.\ thesis and a TODS 2016 paper titled ``Extending the Kernel of a Relational DBMS with Comprehensive Support for Sequenced Temporal Queries'' show how to extend the kernel of PosgreSQL to enable efficient support for the functionality described in the TODS 2000 paper.

\paragraph{Impact and Lessons} Looking back, one may ask what the impact of this work has been. Certainly, the literature suggests that the work has influenced other research in the field, but there has also been impact beyond academia. One highlight is that Teradata put temporal support into their system based on the statement modifier approach, which made them a pioneer in offering temporal support. This was done before ANSI/ISO standardization. Today, Teradata in addition supports the temporal tables and (limited) query language syntax in the standard. Another highlight is that the PostgreSQL implementation described in the TODS 2016 paper is available for anyone to use. A different line of impact is in the area of database design, where national statistics bureaus (e.g., Statistics Denmark) and archives (e.g., Danish National Archives) make use of temporal tables, including bi-temporal tables, when organizing their data. I have been contacted by, and have interacted with, several such entities. While the standards have adopted a language design approach that I think does not scale, and while there is a disconnect between SQL standardization and academia, I do believe that the standard is influenced by advances in temporal database research. For example, the standard supports bitemporal tables: We studied such tables in depth and even coined the term bitemporal.

Finally, I want to make a few points. First, research is often a social and collaborative effort. One should try to work with good colleagues (check!) and try to be a good colleague. Second, it can take decades to achieve societal impact, which is at odds with the increasing dependence on short externally funded projects in order to be able to perform research. Third, the disconnect between stardardization and academia is unfortunate from a societal perspective. Fourth, in research, one often does not quite know where one ends when starting.

%\begin{flushright}
%Christian S.\ Jensen\\
%Aalborg University  
%\end{flushright}

\end{document}
