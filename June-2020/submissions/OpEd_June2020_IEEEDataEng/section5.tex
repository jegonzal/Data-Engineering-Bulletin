\section{Re-imagining Resilience}
With such a complex backdrop, it is more critical than ever to consider the roles for computing in social change ~\cite{abebe2020roles} along with ethical implications~\cite{stoyan, gotterbarn2017acm, bloomberg, EthicalOS, Guidebook} to enable response—responsibly.  Data scientists and data enthusiasts globally have immense potential to light the path towards:

\subsection*{Improved Healthcare Delivery and Accelerated Growth of Immunomics}
The pandemic has surfaced severe inadequacies in healthcare systems worldwide. New approaches to healthcare delivery must incorporate current lessons learned, and context, to address the fundamental needs of local communities and strive for measurable reductions in racial and economic biases. \emph{How might we meet individuals and communities where they are, culturally and metaphorically? How can we imagine and achieve equitable health services across numerous barriers that exist today?} These framing questions should keep us grounded as exciting advances in computational immunomics promise to not only mitigate the effects of new viruses, but also provide novel treatments for autoimmune diseases, cancer, and other conditions. 
 
\subsection*{Reinvigorated Climate Action}
The dramatically altered global patterns of transportation and energy use during the pandemic will allow us to imagine new methods of interacting with each other and our environment. With many international and local borders closed for non-essential activity, and individual confinement extending, we have an opportunity to revisit the impacts of both policy and behavior changes on greenhouse gas emissions, our natural resources, and our clean energy economy~\cite{le2020temporary}. Worldwide, individuals and organizations have a chance to reflect on our choices and climate commitments. Moreover, the pandemic continues to expose the lack of safe, accessible water in underserved rural as well as urban areas, raising opportunities to support equitable infrastructure investment and environmental justice. 
 
\subsection*{New Paradigms of Online Education}
Our concept of how we learn and the future of work may be forever changed. Online learning, once viewed as a less-prestigious option relevant only to a subset of students, is now at least temporarily the norm across the globe, for students from preschool through doctoral programs and in continuing and executive education. Ensuring that \emph{all} students have access to appropriate hardware, software, internet connectivity, and resources to support distance learning is critical~\cite{UCBinterview, Jenna, Pardos}. 
As remote teaching and learning extends, further research on the impacts of adopting new technologies—spanning issues from student privacy, safety, and mental health to the metacognitive aspects of learning, including motivation, resiliency, persistence, and scaffolding, will be needed. We must also be mindful of demands on instructors; training in new methods and logistical support for teachers should become a social priority. Experts, institutions, and authorities who are able to do so should share resources for effective and responsible distance learning widely, following open source models~\cite{unesco}.


