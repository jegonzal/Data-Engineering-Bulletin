\section{Compound Disasters and Interdependencies}

Moreover, as the pandemic extends, the likelihood that additional communities will be impacted by both COVID-19 and compounded disasters—such as wildfires, hurricanes, earthquakes or tornadoes—is increasingly high. The U. S. Federal Emergency Management Agency (FEMA) has published guidance to prepare State, Local, Tribal, and Territorial (SLTT) responders to adapt their response and recovery procedures to this new combination of threats. The COVID-19 Pandemic Operational Guidance for the 2020 Hurricane Season (May 2020) notes the need for response and recovery adaptation that includes virtual damage assessments and significant reliance on the “whole of community” (including the private and non-profit sectors) to maximize their abilities to respond to simultaneously occurring disasters~\cite{FEMAPOG}. 

While in any given year, national-scale agencies managing emergencies routinely respond to multiple disasters simultaneously, SLTT responders often respond to one devastation at a time. Local economies strained by COVID-19 create an inability for communities to truly prepare and protect themselves from all impending hazards. Communities unable to protect themselves may suffer additional strain and damage that makes recovery from the compounded disaster even further from reach.

In an effort to proactively acknowledge limitations, the U.S. National Interagency Fire Coordination Center notes in its May 2020 plans, “In the event of a high disease spread scenario with a high rate of infection, the associated loss of individuals from service will severely tax the ability to maintain an adequate wildfire response, even during a moderately active fire season”. The need for adaptation and re-orienting plans in light of COVID-19 extends to all hazards, as the methodologies and tools SLTT decision makers and responders currently use do not conform to physical distancing guidance and may amplify concerns about responder long-term health.  

Moreover, some communities’ plans and capabilities are not robust to damaged infrastructure or limited internet capacity, rendering strategies for multiple in-depth virtual interactions ineffective at best. Our reflections and how we navigate these challenges promise to impact our communities long after the virus has run its course. 

While the movement to “flatten the curve” offered fairly uniform and straightforward guidance, to shelter in place and physically distance, the road for recovery and the notion of what a “new normal” looks like is riddled with additional interdependencies and tensions. These multifaceted challenges are increasingly being addressed through diverse coalitions, for example, with State-level economic development agencies working alongside university researchers, healthcare institutions, and volunteers to develop and share data, rapidly register products, and mitigate risks as guidance and authorizations evolve~\cite{doi:10.1056/NEJMc2009432,western}.


