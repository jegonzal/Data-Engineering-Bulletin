
\section{Background on Privacy-Sensitive Mobile Contact Tracing}

The key building block for Privacy-Sensitive Mobile Contact Tracing (PS-MCT) is a subtle combination of radio protocols, cryptography, and risk-calculation.  
Phones have a short-range radio, Bluetooth, used to connect to nearby devices.  
To make those connections it periodically broadcasts tiny bits of information. 
The PS-MCT protocols leverages this short-range background broadcast to resolve nearby individuals.\shankari{Is this the final term we decided on? I recall some pushback from David on the term, and I don't remember this as one of the options. Since our current focus is on the idea and not on a working system, it seems like it is a good idea to get the terminology right}


In the Apple-Google Exposure Notification (AGEN) protocol, each phone generates a daily secret key called a Temporary Exposure Key (TEK).
Then every 15 minutes the phone uses the TEK to generate a new 16 byte Rolling Proximity Identifier (RPI). The RPI sequence is generated using a cryptographic hash function, so it does not carry any information about the source individual.  
The current RPI is then continuously broadcast every few hundred milliseconds.
All phones log the RPIs they hear for future exposure analysis.  
Because the RPI is continuously changing, it also cannot be easily tracked.


When someone tests positive they can \textbf{anonymously} publish the daily keys (TEKs) from the days when they were contagious.  
The confirmed positive collection of TEKs is called a Diagnosis Key in the AGEN protocol.  
The Diagnois Keys are published by sharing them with a trusted server which publishes the TEKs for download.
Others can obtain these keys and use the same cryptographic hash functions to recreate the sequence of RPIs. This sequence, combined with some region-specific weights, can determine if they encountered any infected individuals.  
This entire process is accomplished within the Android and iOS operating systems. Government sanctioned apps are only responsible for authenticating infected individuals and, with user permission, publishing the keys. 

It is important to note the distinction between policy and mechanism. The AGEN protocol (and the extensions proposed in this paper) provide a mechanism to detect and notify users about exposure risk, but it is up to public health authorities to define what constitutes an exposure. This distinction is explored in more detail in section \ref{sec:commons}.


