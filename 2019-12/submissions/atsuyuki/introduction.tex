\section{Introduction}
\label{sec:introduction}

Online job platforms are a class of alternative job arrangements that have widely
proliferated in the last few years.
These could be general purpose platforms such as
Amazon Mechanical Turk or CrowdWorks that support a wide variety of tasks entirely online.
Alternatively, they could be specialized platforms such as for ride hailing services
where all tasks are of the same type.
Hybrid platforms such as TaskRabbit matches \emph{labor} providers
such as plumbers with requestors who need such services.
Even platforms such as Etsy could be construed as a job platform
as it allows \emph{goods} providers to sell their products.
Job platforms acts as an intermediary by matching workers with requestors
while taking a cut in the payments.
As much as 36\% of US workers have an alternative work arrangement in some capacity~\cite{mcfeely2018workplace}.
This number is steadily increasing and it is in the realm of possibility
that the vast majority of human workforce will be employed in such arrangements in the near future.

We anticipate a future where there are thousands of online job platforms
for different types of tasks.
Due to their bespoke nature, this will not be a winner takes all.
Despite the impending Cambrian explosion of such platforms,
there has been very limited work on \emph{platform design} --
what are the current pain points of current platforms
and how can they be better designed to handle the future of work?
In this article, we investigate this important problem and make a number of concrete suggestions.

There are three major stakeholders -- workers, requestors and the online job platforms. 
%\scream{Do we need regulators here?. David: I would say so. With a basic role of tax gathering or money laudry monitoring for now, but a role that should expand}
%Saravanan: I would suggest not to include it. we do not discuss it later anyway.
It is important that the platform design takes into account the requirements of each of them.
The current generation of platforms are grappling with how to satisfy each of them.
So far, the efforts are ad-hoc and error prone wherein none of them are satisfied with the status quo.

Current generation of platforms support simple microtasks only.
Support for complex tasks is often rudimentary at best.
We believe that in the future, platforms must be able to support complex knowledge intensive and
collaborative tasks that could involve sophisticated workflows.
When there are a large number of job platforms, it is very important to have formally defined mechanisms to
allow platforms to inter-operate with each other.
This is often beneficial for workers (they could work on multiple platforms at once)
and for requesters (recruit workers from multiple platforms).
A worker should be able to move between platforms without losing
any of the skills/ratings from the original platform.
Similarly, a requester must be able to post a task where workers from multiple platforms are able to contribute.
Achieving this requires tackling a number of technical and policy challenges.
The technical challenges include agreeing upon a predefined schema and API interfaces.
Furthermore, one must identify a class of queries
that allows exchange of information about workers, requesters and tasks.
The regulatory policies must ensure that platforms play well together.

%\scream{The taxonomy given in the following paragraph is not the same as that in Section 4.1}
%We also propose a taxonomy of platforms that form an ecosystem.
%Some examples include
%(i) platforms where the actual tasks are performed,
%(ii) platforms that serve as an online watercooler where workers chat about their tasks and support each other
%(iii) platforms similar to LinkedIn where workers can display their tasks along with credentials for skills to demonstrate their expertise
%(iv) platforms for matching that connect workers to tasks scattered across platforms
%(v) platforms where workers can self-organize and negotiate for employment benefits.
%Interoperability will also bring us different worker-recruitment channels to reach a diverse set of
%workers. Improving interoperability of platforms introduces a larger optimization space for
%solving problems.

\textbf{Paper Outline.}
The rest of the paper is organized as follows.
In Section~\ref{sec:motivation}, we conduct an user study over multiple popular crowdsourcing platforms
whose results indicate the need for platform design.
In Section~\ref{sec:platformDesign}, we enumerate a number of key functionalities that are either missing or not widely supported in the current generation of crowdsourcing platforms.
However, these are also the functionalities that were desired by stakeholders or have the potential to improve their satisfaction.
In Section~\ref{sec:platformInteroperability}, we discuss the important problem of platform interoperability.
As more and more online job platforms are created, it is important that they can interact with each other and we identify the key dimensions 
for enabling this functionality.