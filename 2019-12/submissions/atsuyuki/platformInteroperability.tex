\section{Platform Interoperability}
\label{sec:platformInteroperability}


\begin{figure*}
\small
\begin{tabular}{|p{15mm}|p{3cm}|p{15mm}|p{15mm}|p{15mm}|p{15mm}|p{15mm}|p{15mm}|}
\hline
&&\multicolumn{6}{|l|}{Platforms for$\ldots$}\\
\cline{3-8}
Support for$\ldots$&Functionalities&Providing Jobs&Worker Communication&Worker/ Requester Profiles& Task-Worker Matching&Workflow
management&Worker Training\\
\hline
Worker&Ease of On- and off-boarding&$\bigcirc$&&$\bigcirc$&&&\\
\cline{2-8}
&Learning Skills&&$\bigcirc$&&&&$\bigcirc$\\
\cline{2-8}
&KB for workers&&&&&$\bigcirc$&\\
\cline{2-8}
&Worker Preference Specification&$\bigcirc$&&$\bigcirc$&&&\\
\cline{2-8}
&Ease of choosing tasks&$\bigcirc$&$\bigcirc$&$\bigcirc$&$\bigcirc$&&\\
\hline
Requester&Task Requirement Specification&$\bigcirc$&&&&&\\
\cline{2-8}
&Complex Workflow&&&&&$\bigcirc$&\\
\cline{2-8}
&Sophisticated Assignment Algorithm&&&&$\bigcirc$&$\bigcirc$&\\
\cline{2-8}
&AI Worker Support&&&$\bigcirc$&$\bigcirc$&$\bigcirc$&\\
\cline{2-8}
&Total Optimization&&&&&$\bigcirc$&$\bigcirc$\\
\cline{2-8}
&Algorithm Butiques&&&&$\bigcirc$&$\bigcirc$&\\
\cline{2-8}
&Total Optimization&&&&&$\bigcirc$&$\bigcirc$\\
\cline{2-8}
&Bespoke Platforms&$\bigcirc$&&$\bigcirc$&&&$\bigcirc$\\
\cline{2-8}
&Workflow and AI Worker Library&&&$\bigcirc$&&$\bigcirc$&\\
\hline
\end{tabular}
\caption{Relationship between functionalities and platforms for FoW. 
This suggests that cooperation between different platforms will be the key to make use of the full potential of ecosystem of FoW platforms. 
%The circle/triangle can be replaced with terms that explain what is provided by the platform.
}
\label{fig:functionsandplatforms}
\end{figure*}

In the previous section, we discussed the various functionalities that could be added to the platform to improve it.
In this section, we discuss how additional benefits could be obtained by enabling interoperability between platforms.
A vast ecosystem has emerged around online job platforms.
We begin by creating a taxonomy of such platforms and discuss how these functionalities could be improved and integrated.


\subsection{Taxonomy of Online Job Platforms and Ecosystems}





So far, the prior work has extensively discussed online job platforms such as AMT or CrowdWorks.
However, there is a thriving ecosystem built around these platforms.
They often add features missing in the original platforms or provide value added services.
In order to improve platform design, it is important to understand this ecosystem.


\textbf{Job Platforms.}
For the sake of completeness, we include online job platforms in the taxonomy.
These are the platforms such as AMT, CrowdWorks and others where requesters post tasks
and workers complete them \emph{online}.
These could be generic platforms where a wide variety of tasks are performed or
specific ones such as ride-hailing that perform a single task.

\textbf{Platforms for Worker Communication.}
Most of the current platforms do not support communications between workers.
Some times, this is desirable to avoid inter-worker collusion.
Most of the time, this is very limiting as collaborative tasks become more and more important.
Furthermore, there are also many scenarios where workers might want to communicate.
These include providing feedback about requesters, providing tips and so on.
So there is an emerging set of platforms to enable such communication.
The most popular of these are Turkopticon~\cite{irani2013turkopticon} and TurkerNation (which closed in 2018).
Turkopticon has almost 500K reviews of around 50K requesters.
It also provides granular way to provide requestor feedback on fairness, pay quality and speed of payment.
While these are a good start, both Turkopticon and TurkerNation were specific to AMT.
It is important that most of these platforms provide a mechanism for worker communication.
This could be done within each platform or by providing a generic website where workers from different platforms can congregate.

\textbf{Platforms for Persistent Worker/Requester Profiles.}
Currently, each online job platform has an internal mechanism for measuring worker/requestor reputation.
This could be as simple as approval rate for workers and requesters.
It could also be more granular such as how many tasks a worker has completed for each skill.
It is important to have persistent profiles for both workers and requesters so that
they can build their reputations based on their entire body of work.
Currently, if an experienced worker moves from one platform to another,
she has to start from scratch.
Hence, it is important to have an independent website where workers/requesters
can create a profile and share all relevant information in a verified manner.

\textbf{Platforms for Matching Workers and requesters.}
The explosion of online job platforms makes it harder for both workers and requesters.
Productive workers and fair requesters are often scattered across platforms.
Each worker has to login to multiple platforms to find useful work that quickly becomes tedious.
It is desirable to have platforms that could match workers and requesters across multiple platforms.
For example, the worker could specify task preferences and the matching platform will scour
other online job platforms and notify the worker when relevant tasks become available.
This could also benefit requesters by finding productive workers across platforms.
While there are some preliminary effort such as by CrowdWorks or Figure Eight
that act as a layer on top on AMT, there is still lot of work to be done.

\textbf{Towards Better and Fairer Crowdsourcing Platforms.}
Recently, there has been increasing interest in improving crowdsourcing.
Some of these efforts include providing guidelines for requesters for specific platforms.
Popular examples include Dynamo Guidelines~\cite{salehi2015we} for academic requesters and FairCrowdWork.org.
These often help well-meaning requesters to treat workers properly
by providing clear instructions about the task and fair pay.
FairCrowdWork.org provides advice to workers based on their rights.
They have extensive survey involving 95 questions that allows workers to rate various crowdsourcing platforms.
Specifically, they quizzed workers on 8 topics including demographics, work experience in a platform, payment details, communication with requesters and platforms, reliability of the platform, quality of available tasks and miscellaneous information.
Based on the response of the workers, each platform is graded in a scale between 1-5.




\subsection{Platform Interoperability}
Given the plethora of platforms, it is important they are interoperable to ensure that
workers and requesters have a wide variety of options.
Currently, interoperability is nearly non-existent.
However, we believe that this will become more and more important
as an increasing proportion of workers join these online job platforms.

There are a number of dimensions in which interoperability must be achieved.
Workers should be able to move between platforms without losing any of their skills/ratings etc.
This will require the creation of a unified schema of skills. This could be different for each domain.
Once this is done, each platform has to fill the worker skills according to this schema.
Of course, each platform could have their own additional proprietary list of metric.
Nevertheless, it must be possible to create a worker ``profile'' that gives a holistic
perspective of the worker skills in that domain regardless of how many platforms the
worker is involved with.
Interoperability must also be possible from a requestor perspective. Given a task, it must be
possible to recruit workers from multiple platforms.
For example, if a requestor needs 10 qualified experts, it must be possible to
recruit 4 from Platform 1, 5 from Platform 2 and 1 from a third one.
Similarly, it is important to create a requester ``profile'' that is persistent across multiple platforms
so that the worker has a holistic perspective of the requester such as the rate of rejection of tasks. 

We can see that achieving interoperability is quite challenging.
This requires a commonly agreed standard of skills, which,
due to the richness and specialties of the many disciplines that
exist, is probably difficult to achieve.
For platforms to exchange information, one must design a common, unifying API
that will fit for all types of applications.

\textbf{Standardized API and Schema:} 
The diversity of the crowdsourcing and FoW platforms makes standardization much challenging.
However, it is possible that specific components such as task creation, worker selection, truth inference, worker skill estimation
could be more amenable for standardization.
The creation of the standard API would boost the adoption of crowdsourcing by bringing in more enterprises.
Entrepreneurs could also create innovating applications for specific crowdsourcing tasks.
This is akin to the explosion of smartphone apps once iOS and Android provided a standardized API functionality.
For example, one could identify an important tasks such as Entity Resolution (finding tuples that represent the same real-world entity)
and use the API of the platforms to recruit the crowd, perform the tasks and provide the results. 
There will also be standardization in terms of how metadata about workers and tasks are stored \cite{DBLP:books/daglib/0031391,DBLP:journals/www/SchallSP14}.

\textbf{Interchange Format:} There is a need to have a pre-defined interfaces and Schema for data
exchanges between platforms. It should provide a simple agreed model that has enough
expressive power to achieve total optimization. 
At the minimum, one must be able to transfer the human factors of workers (such as skill, past tasks, etc)
and that requesters (task approval rate, etc). 
This allows both these types of users to migrate to a different platform without being locked in. 

\textbf{Interchange Queries:} When data is not large and is allowed to be exported, platforms can exchange the whole data in the interchange format,
but there are cases where this is not a practical solution. An effective approach is to  develop common sets of interchange queries so that we can identify only small subsets of data to be exchanged.



