\documentclass[11pt]{article} 

\usepackage{deauthor,times,graphicx}
%\usepackage{url}
\usepackage{hyperref}


\begin{document}
We envision the Future of Work (FoW) to be a place where humans are empowered with the ability to rely on AI machines in an on-demand fashion, with ability to juggle diverse job opportunities that provide intellectual self-actualization, and enhance their capabilities by continuous knowledge and skill acquisition with a variety of onboarding and training tools. We recognize that the current view of ``humans-in-the-loop'' tends to see humans as machines, robots, or low-level agents used or exploited in the service of broader AI goals. Our hope is that people in their workplace in the future should be treated as fully human, with respect and dignity with the right to productive employment and the goal of bringing them back in the ``frontier'' of ``humans-in-the-loop'' systems. Such an environment will allow everyone everywhere to get a job online and offline, train for a new job, and get help from a mix of humans and AI machines. Ensuring portability between FoW marketplaces, and guaranteeing the protection of workers' rights, will play a major role in providing a rewarding and safe work environment to all. Existing platforms must rethink their design to empower humans and be at the frontier of FoW.

Today, humans' relationship to work is changing as online job platforms are blurring the boundaries between physical and virtual workplaces. That is witnessed in freelancing platforms such as CrowdWorks in Japan, TaskRabbit and Fiverr in the USA, and Qapa and MisterTemp in France, crowdsourcing platforms such as Crowd4U in Japan, gMission in Hong Kong, Wirk, and Prolific Academic in Europe, Amazon Mechanical Turk and Figure Eight in the USA, and platforms that provide help with entrepreneurship such as de Asra in India. Prospective employees can find temporary jobs in the physical world (e.g., a plumber, an event organizer, a gardener, can offer their jobs online), or in the form of virtual gigs (e.g., logo design, web programmer). Job providers can hire one or many individuals to achieve a task. The same person can take on those roles at any point in time. An employer can be a regular citizen who needs to hire a plumber, a social scientist needed to conduct population studies to verify some theories, a data scientist needing to validate a new algorithm, a domain expert seeking to verify how much interest a new product generates. The diversity of needs has given rise to a variety of platforms, all of which act as intermediaries between job providers and job seekers. Platforms differ in their ability to manage physical and virtual jobs, in their support for onboarding, socializing, training, and credentialing for employees, in automating the matching between jobs and workers. They also differ in the tools they offer to workers and job providers to express needs and requirements, and in their compliance with labor-related regulations and their handling of ethical concerns. 

In this issue, to follow the trend that FoW will be increasingly technology-driven and will have the potential to bring human concerns to the center of the design and deployment of job platforms,  we discuss the raised intellectual challenges.

The first paper addresses the challenge related to the ability to understand different types of human roles in future jobs, modeling the inherent uncertainty in human behavior by understanding their evolving characteristics and be able to propose jobs to them by adapting to their changing perceptions and needs.

The second paper addresses the challenge in the development of appropriate interaction methodologies and support for the new kind of workplace. This includes social processes relating workers, requesters and platform managers like onboarding, socializing, recognition and training, as well as ways to communicate and delegate work between humans and machines.

The third paper addresses the challenge on how to design platforms that maximize the satisfaction of various stakeholders.

The fourth paper addresses the challenge on design benchmark and metrics to measure computing capabilities as well as human aspects such as satisfaction, human capital advancement, and equity.

The last paper discusses various of ethics issues raised in FoW design and architecture, including privacy, compensation mechanisms and fairness.

We  would like to thank all the authors for their insightful contributions. 
\end{document}


