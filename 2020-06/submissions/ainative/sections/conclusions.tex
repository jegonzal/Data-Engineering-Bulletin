%!TEX root = ../main.tex

\section{CONCLUSION}
\label{sec: conculstion}

We proposed an AI-native database \oursys, which not only utilizes AI techniques to enable self-configuring, self-optimizing, self-monitoring, self-healing, self-diagnosis, self-security, and self-assembling, but also provides in-database AI capabilities to lower the burden of using AI. We categorized AI-native databases into five levels, AI-advised, AI-assisted, AI-enhanced, AI-assembled, and AI-designed. We also discussed the research challenges and provided opportunities in designing an AI-native database.  



%which includes five levels to benefit from DB, AI and new computing powers.  First, we packed AI tools into the database to achieve auxiliary optimization. Second, we implanted them into the DB kernel to improve efficiency. Third, we reshaped DB kernel based on AI tech and designed a unified engine to provide DB and AI services at the same time. Fourth, we designed a heterogeneous computing architecture to better support operations of both DB and AI. Finally, we integrated AI into the whole life cycle of database to achieve a real AI-native database. And in the end we have also discussed some challenges in deploying \oursys.


%We proposed a query-aware database tuning system \oursys with a deep reinforcement learning (DRL) model. \oursys featurized the SQL queries by considering rich features of the SQL queries. \oursys fed the query vectors  into the DRL model to  dynamically choose suitable configurations. Our DRL model used the the actor-critic networks to find optimal configurations according to both the current and predicted database states. Our tuning system can support query-level, workload-level and cluster-level database tuning. We also proposed a query clustering method using deep learning model to enable cluster-level tuning. Experimental results showed that \oursys achieved high performance and outperformed the state-of-the-art tuning methods. 

