\section{Properties of a nutritional label}
\label{sec:prop}

The database and cyberinfrastructure communities have been studying systems and standards for metadata, provenance, and transparency for decades~\cite{op,DBLP:journals/concurrency/MoreauLA08}. We are now seeing renewed interest in these topics due to the proliferation of data science applications that use data opportunistically.  Several recent projects explore these concepts for data and algorithmic transparency, including the Dataset Nutrition Label project~\cite{DBLP:journals/corr/abs-1805-03677}, Datasheets for Datasets~\cite{DBLP:journals/corr/abs-1803-09010}, and Model Cards~\cite{DBLP:conf/fat/MitchellWZBVHSR19}. 
%We will discuss these methods further in Section~\ref{sec:conc}, and note here that a
All these method rely on manually constructed annotations.  In contrast, our goal is to {\em generate labels automatically or semi-automatically}.

To differentiate a nutritional label from more general forms of metadata, we articulate several properties:
\begin{itemize}
\item {\bf Comprehensible}: The label is not a complete (and therefore overwhelming) history of every processing step applied to produce the result.  This approach has its place and has been extensively studied in the literature on scientific workflows, but is unsuitable for the applications we target.  The information on a nutritional label must be short, simple, and clear.

\item {\bf Consultative}: Nutritional labels should provide actionable information, rather than just descriptive metadata.  For example, universities may invest in research to improve their ranking, or consumers may cancel unused credit card accounts to improve their credit score.

\item {\bf Comparable}: Nutritional labels enable comparisons between related products, implying a standard. % We will revisit this point in Section~\ref{sec:conc}.
The IEEE is developing a series of ethics standards, known as the  IEEE P70xx series, as part of its Global Initiative on Ethics of Autonomous and Intelligent Systems.\footnote{\url{https://ethicsinaction.ieee.org/}}  These  standards include ``IEEE P7001: Transparency of Autonomous Systems'' and  ``P7003: Algorithmic Bias Considerations''~\cite{DBLP:conf/icse/KoeneDS18}.  The work on nutritional labels is synergistic with these efforts.

\item {\bf Concrete}: The label must contain more than just general statements about the source of the data; such statements do not provide sufficient information to make technical decisions on whether or not to use the data.  
\end{itemize}
 
Data and models are chained together into complex automated pipelines — computational systems ``consume'' datasets at least as often as people do, and therefore also require nutritional labels!  We articulate additional properties in this context:

\begin{itemize}
\item {\bf Computable}: Although primarily intended for human consumption, nutritional labels should be machine-readable to enable specific applications: data discovery, integration, automated warnings of potential misuse.  

\item {\bf Composable}: Datasets are frequently integrated to construct training data; the nutritional labels must be similarly integratable.  In some situations, the composed label is simple to construct: the union of sources. In other cases, the biases may interact in complex ways: a group may be sufficiently represented in each source dataset, but underrepresented in their join.  
\item {\bf Concomitant}: The label should be carried with the dataset; systems should be designed to propagate labels through processing steps, modifying the label as appropriate, and implementing the paradigm of transparency by design.
\end{itemize}


