\documentclass[11pt]{article} 

\usepackage{deauthor,times,graphicx}
%\usepackage{url}
\usepackage{hyperref}



\begin{document}


Databases have played a very important role in many applications and been widely deployed in many fields. However, traditional database design is still based on empirical methodologies and specifications,  and require heavy human involvement to tune and maintain the databases.   Recently there are many attempts that use AI techniques to optimize database, e.g., learned index, learned cost estimation, learned optimizer and learning-based database knob tuning. In this issue, we discuss (1) how to utilize AI techniques to improve database and (2) how to utilize database techniques to benefit AI and provide in-database capabilities. 

The first paper integrates data preparation into data analysis and adapts data preparation to each workload which can minimize response times.

The second paper discusses how to provide machine learning capabilities inside a database, which is an FPGA-enabled database engine incorporating FPGA-based machine learning operators into a main memory, columnar DBMS. 


The third paper presents two engineering approaches for integrating ML agents in a DBMS. The first is to build an external tuning controller that treats the DBMS as a black-box. The second is to integrate the ML agents natively in the DBMS architecture.  

The fourth paper proposes the construction of an engine, a Data Alchemist, which learns how to blend fine-grained data structure design principles to automatically synthesize brand new data structures.

The fifth paper discusses the AutoML problem and proposes MILE, an environment where humans and machines together drive the search for desired ML solutions. 

The last paper proposes AI-native database. On one hand, it integrates AI techniques into databases to provide self-configuring, self-optimizing, self-healing, self-diagnosis, self-monitoring, self-security and self-assembling capabilities for databases. On the other hand, it can enable databases to provide AI capabilities using declarative languages, in order to lower the barrier of using AI.  




I would like to thank all the authors for their insightful contributions. I hope you enjoy reading the papers. 


\end{document}


