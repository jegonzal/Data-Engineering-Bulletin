%!TEX root = ../main.tex

\begin{abstract}
In big data era, database systems face three challenges. Firstly, the traditional empirical optimization techniques (e.g., cost estimation, join order selection, knob tuning) cannot meet the high-performance requirement for large-scale data, various applications and diversified users. We aim to  design learning-based techniques to make database more intelligent. Secondly, many database applications require to use AI algorithms, e.g., image search in database. We can embed AI algorithms into database, utilize database techniques to accelerate AI algorithms, and provide AI capability inside databases. Thirdly, traditional databases focus on using general hardware (e.g., CPU), but cannot fully utilize new hardware (e.g., ARM, AI chips). Moreover, besides relational model, we can utilize tensor model to accelerate AI operations. Thus, we need to design new techniques to make full use of new hardware. 

To address these challenges, we design an AI-native database. On one hand, we integrate AI techniques into databases to provide self-configuring, self-optimizing, self-healing, self-protecting and self-inspecting capabilities for databases. On the other hand, we can enable databases to provide AI capabilities using declarative languages, in order to lower the barrier of using AI.  

In this paper, we will introduce the five levels of AI-native databases and provide the open challenges of designing an AI-native database. We will also take autonomous database knob tuning, deep reinforcement learning based optimizer, machine-learning based cardinally estimation, and autonomous index/view advisor as examples to showcase the superiority of AI-native databases. 
\end{abstract}




%Incorporate features like Intelligence, Integration and Heterogeneity into database design are of great significance. Existing databases have several limitations. First, they are not intelligent. Databases are unaware of workload type and try to use one mode to fit all. Second, they are not integrated. Databases are usually based on single data model and cannot support integrated data analysis (IDA). Third, they are not heterogeneous. Database mainly rely on CPU and cannot efficiently utilize different computing powers. 

%To address these problems, we propose an AI-Native database \oursys , which includes five developing stages to benefit from DB, AI and new computing powers.  First, we pack AI tools into the database to achieve auxiliary optimization. Second, we implant them into the DB kernel to improve efficiency. Third, we reshape DB kernel based on AI tech and provide unified engine to provide both DB and AI services. Fourth, we deploy heterogeneous computing architecture to better support operations of both DB and AI. Finally, we integrate AI into the whole life cycle of database to achieve a real AI-Native database.
